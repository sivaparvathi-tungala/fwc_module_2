\documentclass{article}
\usepackage{amssymb,amsfonts,amsthm,amsmath}
\usepackage{enumitem}
\providecommand{\pr}[1]{\ensuremath{\Pr\left(#1\right)}}
\providecommand{\cbrak}[1]{\ensuremath{\left\{#1\right\}}}
\newcommand{\solution}{\noindent \textbf{Solution: }}
\newcommand*{\permcomb}[4][0mu]{{{}^{#3}\mkern#1#2_{#4}}}
\newcommand*{\perm}[1][-3mu]{\permcomb[#1]{P}}
\newcommand*{\comb}[1][-1mu]{\permcomb[#1]{C}}
\setlist[enumerate]{font=\small\bfseries}
\renewcommand\thefootnote{\textcolor{black}{\arabic{footnote}}}
\def\inputGnumericTable{}
\usepackage[latin1]{inputenc}
\usepackage{fullpage}
\usepackage{color}
\usepackage{array}
\usepackage{longtable}
\usepackage{calc}
\usepackage{multirow}
\usepackage{hhline}
\usepackage{ifthen}

\begin{document}

\title{PROBABILITY}
\author{\Large T SIVA PARVATHI - FWC22089}
\date{}

\maketitle
\begin{enumerate}[label=13.\arabic{enumi}.\arabic{enumii}]%,ref=\thesection.\theenumi.\theenumi]
\numberwithin{equation}{enumi}
%\numberwithin{table}{enumi}
\setcounter{enumi}{3}
\setcounter{enumii}{5}

\item \footnote{Read question numbers as (CHAPTER NUMBER).(EXERCISE NUMBER).(QUESTION NUMBER)}
Find the probability distribution of the number of successes in two tosses of a die, where a success is defined as
\begin{enumerate}
\item number greater than 4
\item six appears on at least one die
\end{enumerate}

\solution
Given that a die tossed two times,

Consider each trial results in success or failure. Let $X_i$ where $i = 1,2$ be the random variables representing the outcome for each die toss. 
\begin{table}[h]\centering
	\input{dtable.tex}
	 \caption{Variable Description}\label{tab:}
\end{table}

$p$ and $q$ are the probability of success and failure respectively.
\begin{align}
& p_1 = \frac{1}{3}&             
\\            
& q_1 = 1 - p_1 = \frac{2}{3}&      
\end{align}
Similarly,
\begin{align}
& p_2 = \frac{1}{6}&             
\\            
& q_2 = 1 - p_2 = \frac{5}{6}&      
\end{align}
\begin{enumerate}
\item number greater than 4

In $n$ Bernoulli trials with $k$ success and $(n - k)$ failures, the probablity of $k$ success in $n$- Bernoulli trials can be given as\\
\begin{align}
\pr{X_i=k}  &= 
\begin{cases}
\comb{n}{k} p^{k}q^{n-k} & 0 \le k \le n
\\
0 & \text{otherwise}                
\end{cases}
\end{align}
where, $n = 2$
\begin{align}
X&=X_1+X_2\\
p_X(k)&=\comb{n}{k}{p_1}^{k}{q_1}^{n-k}, 0\le k\le 2,n=2
\end{align}
Probability distribution of getting number greater than 4 is,
\begin{align}
  p_X(k) =
    \begin{cases}
      \frac{4}{9}, &  k = 0\\
      \frac{4}{9}, & k = 1\\
      \frac{1}{9}, & k = 2
    \end{cases}       
\end{align}
\item six appears on at least one die

In $n$ Bernoulli trials with $k$ success and $(n - k)$ failures, the probablity of $k$ success in $n$- Bernoulli trials can be given as\\
\begin{align}
\pr{X_i=k}  &= 
\begin{cases}
\comb{n}{k} p^{k}q^{n-k} & 0 \le k \le n
\\
0 & \text{otherwise}                
\end{cases}
\end{align}
where, $n = 2$
\begin{align}
X&=X_1+X_2\\
p_X(k)&=\comb{n}{k}{p_2}^{k}{q_2}^{n-k}, 0\le k\le 2,n=2
\end{align}
Probability distribution of getting six on atleast one die is,
\begin{align}
  p_{X}(k) =
    \begin{cases}
      \frac{25}{36}, &  k = 0\\
      \frac{10}{36}, & k = 1\\
      \frac{1}{36}, & k = 2
    \end{cases}       
\end{align}
\end{enumerate}
\end{enumerate}
\end{document}