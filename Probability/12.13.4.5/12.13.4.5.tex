\documentclass{article}
\usepackage{amssymb,amsfonts,amsthm,amsmath}
\usepackage{enumitem}
\providecommand{\pr}[1]{\ensuremath{\Pr\left(#1\right)}}
\providecommand{\cbrak}[1]{\ensuremath{\left\{#1\right\}}}
\newcommand{\solution}{\noindent \textbf{Solution: }}
\newcommand*{\permcomb}[4][0mu]{{{}^{#3}\mkern#1#2_{#4}}}
\newcommand*{\perm}[1][-3mu]{\permcomb[#1]{P}}
\newcommand*{\comb}[1][-1mu]{\permcomb[#1]{C}}
\setlist[enumerate]{font=\small\bfseries}
\renewcommand\thefootnote{\textcolor{black}{\arabic{footnote}}}
\def\inputGnumericTable{}
\usepackage[latin1]{inputenc}
\usepackage{fullpage}
\usepackage{color}
\usepackage{array}
\usepackage{longtable}
\usepackage{calc}
\usepackage{multirow}
\usepackage{hhline}
\usepackage{ifthen}

\begin{document}

\title{PROBABILITY}
\author{\Large T SIVA PARVATHI - FWC22089}
\date{}

\maketitle
\begin{enumerate}[label=13.\arabic{enumi}.\arabic{enumii}]%,ref=\thesection.\theenumi.\theenumi]
\numberwithin{equation}{enumi}
%\numberwithin{table}{enumi}
\setcounter{enumi}{3}
\setcounter{enumii}{5}

\item \footnote{Read question numbers as (CHAPTER NUMBER).(EXERCISE NUMBER).(QUESTION NUMBER)}
Find the probability distribution of the number of successes in two tosses of a die, where a success is defined as
\begin{enumerate}
\item number greater than 4
\item six appears on at least one die
\end{enumerate}

\solution
Given that a die tossed two times,
\begin{table}[h]\centering
	\input{dtable.tex}
	 \caption{Random Variables(RV) X and Y}\label{table:1}
\end{table}

\begin{enumerate}
\item number greater than 4

when we throw a die twice, there are three cases

X=0 means no number greater than 4

X=1 means 1 number greater than 4 

X=2 means 2 numbers greater than 4 
\begin{align}
  p_X(k) = \pr{X=k} =
    \begin{cases}
      \frac{4}{9},  & \text{ $k =$ 0}\\
      \frac{4}{9}, & \text{  $k =$ 1}\\
      \frac{1}{9}, & \text{  $k =$ 2}
    \end{cases}       
\end{align}
\item six appears on at least one die
\end{enumerate}

there are two cases,

Y=0 means number 6 doesnot appear at all 

Y=1 means number 6 appears atleast on one die 
\begin{align}
\pr{Y=1}=\frac{11}{36}\\
\pr{Y=0}=1-\pr{Y=1}=\frac{25}{36}
\end{align}
\begin{align}
  p_Y(k) = \pr{Y=k} =
    \begin{cases}
      \frac{25}{36},  & \text{ $k =$ 0}\\
      \frac{11}{36}, & \text{  $k =$ 1}
    \end{cases}       
\end{align}
\end{enumerate}
\end{document}